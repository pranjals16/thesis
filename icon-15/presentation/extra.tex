\section[Transcryptfs in action]{Transcryptfs in action}
\subsection{}

 \frame{\frametitle{Mount}
    \begin{beamerboxesrounded}[shadow=true]{}
  % Transcryptfs is implemented as a stacked filesystem. Therefore, it doesn’t create a filesystem directly on the hardware like disk. It mounts on top of already existing file system
  	\begin{enumerate}
  		\item At the first mount, filesystem related metadata is generated and written in a hidden file named \emph{transcryptfs\_fsk\_hash} stored in the lower filesystem
  		\item \textbf{mount.transcryptfs} utility asks administrator for the mount passphrase and the location of CA’s public-key certificate. These are passed to kernel
  		\item CA’s public key is read and stored into an in-memory cryptographic structure in kernel space
  		\item The mount passphrase is extracted and a key is derived from it using PBKDF2 (takes 20 bytes of random salt)
  		\item A random 16 bytes FSK is generated in kernel space
  		\item FSK is encrypted with the derived key and written in the hidden file, \emph{transcryptfs\_fs\_hash}
  		\item An MD5 hash of FSK is also stored along with the salt and the encrypted FSK in the hidden file as filesystem metadata
  	\end{enumerate}
    \end{beamerboxesrounded}
  }
  \frame{\frametitle{Subsequent Mount}
    \begin{beamerboxesrounded}[shadow=true]{}
  % Transcryptfs is implemented as a stacked filesystem. Therefore, it doesn’t create a filesystem directly on the hardware like disk. It mounts on top of already existing file system
  	\begin{enumerate}
  		\item \textbf{mount.transcryptfs} utility asks for the mount passphrase and the location of CA’s public-key certificate and passes it to kernel space
  		\item CA’s public key is read and stored into an in-memory cryptographic structure in kernel space
  		\item The stored salt, encrypted FSK and FSK's MD5 hash is read from the filesystem's metadata stored in the hidden file
  		\item Kernel uses the salt to derive the key using PBKDF2 as before and decrypts the encrypted FSK to obtain FSK
  		\item An MD5 hash of the FSK is calculated and matched with the FSK’s MD5 hash obtained from the filesystem’s metadata. If the two hashes match each other, mount is successful.
  	\end{enumerate}
    \end{beamerboxesrounded}
  }
  \frame{\frametitle{Create}
    \begin{beamerboxesrounded}[shadow=true]{}
      \begin{enumerate}
      \item A new FEK is randomly generated by the kernel
      \item FEK encrypted with the FSK to obtain the \emph{blinded}FEK
      \item Transcryptfs searches for the user's public-key certificate in its certificate list. If not found, it obtains from \emph{Transcryptd}
      \item The certificate is verified inside kernel using CA's public key
      \item The public-key is used to encrypt the \emph{blinded}FEK to form user's authentication token. This token is used to give this user access to the file.
      \item The token is written at the beginning of the file in the form of token entry and forms a part of file's metadata.
      \begin{center}
      \emph{token\_entry = uid $|$ cert-id $|$ token-size $|$ token}
      \end{center}
      \end{enumerate}
    \end{beamerboxesrounded}
  }
  \frame{\frametitle{Open}
    \begin{beamerboxesrounded}[shadow=true]{}
      \begin{enumerate}
   \item Kernel extracts \emph{SID} from the session keyring of the user who has issued the open system call
   \item It finds the \emph{session entry} corresponding to the \emph{SID} in its \emph{session list}. If the \emph{session entry} is not found, the user's access is denied with $-EIO$ return value
   \item The \emph{session entry} contains the session's credentials stored at the time of user's login: \emph{UID} and $SK_{fs\_pks}$
   \item \emph{UID} is extracted from the \emph{session entry} and matched with the \emph{UID} associated with the user's process who has issued the open system call.
   \item If the two \emph{UIDs} don't match, we assume that the user's process has been masqueraded by an attacker and the access is denied with $-EIO$ return value.
  \end{enumerate}
    \end{beamerboxesrounded}
  }
  \frame{\frametitle{Open}
    \begin{beamerboxesrounded}[shadow=true]{}
      \begin{enumerate}
      \item Otherwise, token decryption request is sent to AuthServer via Transcryptd, $E^{S}_{SK_{fs\_pks}}(token)$
      \item \emph{AuthServer} decrypts the encrypted token using the session key $SK_{fs\_pks}$
      \item It then decrypts the token using user's private-key to obtain \emph{blinded}FEK
      \item \emph{blinded}FEK is encrypted and passed to kernel via Transcryptd
      \item Kernel decrypts the packet to obtain the \emph{blinded}FEK. The \emph{blinded}FEK is decrypted with \emph{FSK} to obtain FEK
      \item The FEK is stored in the file's cryptographic context in memory and later used to decrypt or encrypt file contents at the time of read or write
  \end{enumerate}
    \end{beamerboxesrounded}
  }
  \frame{\frametitle{Read / Write}
    \begin{beamerboxesrounded}[shadow=true]{}
      \begin{itemize}
       \item Files written into the Transcryptfs mounted filesystem are encrypted and stored in the lower filesystem on top of which the Transcryptfs is mounted
       \item When files are read, the encrypted file contents stored in the lower filesystem are decrypted and kept in the page cache
       \item Further edits are made in the page cache itself
       \item Reduces overload of encryption/decryption every time file is written or read
      \end{itemize}
  \end{beamerboxesrounded}
  }
  \frame{\frametitle{Add User}
    \begin{beamerboxesrounded}[shadow=true]{}
      \begin{enumerate}
      \item  \textbf{\emph{setfacl}} utility
      \item The utility asks for the path of the file to which the user is to be granted access and the UID of the user to whom this access is beign granted
      \item It passes these values to kernel
      \item Kernel checks for any masquerading attack
      \item If not, kernel extracts the token and sends it to AuthServer to obtain \emph{blinded}FEK
      \item Searches for the new user's public-key certificate in its \emph{cert list}. If not found, it gets the certificate from Transcryptd
      \item The public-key is used to form the token. The token entry corresponding to the new user's token is added in the file's metadata
  \end{enumerate}
    \end{beamerboxesrounded}
  }

