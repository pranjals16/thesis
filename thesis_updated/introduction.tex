\chapter{Introduction}
Language representation is a very crucial aspect to perform various NLP tasks and has been looked into great detail in recent times. Language representation models have fallen into broadly two categories: ones which require hand-trained language databases such as treebanks (e.g.,\cite{Socher:13}) and ones which are language agnostic and work on raw corpora (e.g., LDA, BOW, SkipGram, NLM, etc.). Liu \cite{Liu:15} compare various language agnostic models for topic modeling. We have tried to present a work which is language independent and which is enriched with vital properties of other models. This ensures that our model is effective as well as accurate in various languages. The world class results in English clearly indicate the efficacy of our approach and improvements in Hindi depict the deficiency in other models which were used earlier.

The large decrease in computational complexity of recent models~\cite{Mikolov:13a,Mikolov:13b,Le:14} has led us to experiment with various factors and parameters helping us in providing justification and validation to arguments which were not experimentally justified but claimed on the basis of intuition.

Language independent models such as LDA  and BOW have been quite effective since long time. Variants of BOW such as tf-idf had changed the perception of researchers towards these models when they were proved effective in various NLP tasks. LDA was able to model inter and intra documental statistical and relational structure quite well overcoming the drawbacks of BOW. But the semantic and syntactical dependencies were still ignored. After the introduction of neural language vector models,  NLP saw a huge diversion in representation of words and documents.

Vector models for individual words are obtained via distributional learning, the mechanisms for which varies from document-term matrix factorization~\cite{Landauer:97}, various forms of deep learning~\cite{Collobert:08,Turian:10,Socher:13}, optimizing models to explain co-occurrence constraints ~\cite{Mikolov:13a,Pennington:14},etc. Once the word vectors have been assigned, similarity between words can be captured via cosine distances. The same models have been extended~\cite{Le:14} with new variables to build vector models for sentences and documents. These models include the essence of individual words as well as their relative order in terms of sentence vector which was earlier absent in word vectors. The advantage of these approaches is that they can capture both the syntactic and the semantic similarity between words/documents in terms of their projections onto a high-dimensional vector space; further, it seems that one can tune the privileging of syntax over semantics by using local as opposed to large contexts~\cite{Huang:12}.

There has been another problem which has attracted the attention of Indian grammarians for over a period of nearly a millennium. They have been trying to find whether sentence meaning accrues by combining word meanings, or whether words gain their meanings based on the context they appear in \cite{Matilal:90}.  The former position, that meaning is {\em compositional}, has been associated with the fregean enterprise of semantics, whereas recent models, building on large corpora of text (and associated multimedia) a large degree of success has accrued to models that attempt to model word meaning based on their linguistic context (e.g. ~\cite{Landauer:97}). The latter line has resulted in strong improvements in several NLP tasks using word vectors~\cite{Collobert:08,Turian:10,Mikolov:13a,Socher:13}.

For resource-poor languages, these approaches have the added lure that many of these methods are completely unsupervised and work directly with large raw text corpora, thus avoiding contentious issues such as deciding on a POS-tagset, or expensive human annotated resources such as treebanks.  For Indian languages which are highly inflected, stemming or identifying the lemma is another problem
which such models can overcome, provided the corpus is large enough. Nonetheless, this approach remains under-explored for Indian languages. At the same time, it must be noted that many approaches seek to improve their performance by combining POS-tags and even parse tree structures into the models for higher accuracies in specific tasks~\cite{Socher:13}. 

One problem in this approach is that of  combining the word vectors into larger phrases. In past work, inverse-similarity weighted averaging appears to work to some extent even for complex tasks such as essay grading \cite{Landauer:03}, but multiplicative models (based on a reducing the tensor products of the vectors) appears to correlate better with human judgements~\cite{Mitchell:08,Socher:13}.
Another complexity in composition is that composing words across phrasal boundaries are less meaningful than composing them within a phrase - this has led to models that evaluate the nodes of a parse tree, so that only coherent phrases are evaluated~\cite{Socher:13}. The results reported here, are based on applying the Skip-Gram  model~\cite{Mikolov:13b} to Hindi. 

\cite{Turney:10} give a detailed overview of various vector space models and their composition. A surprising event in Information Theory has higher information content than an expected event (Shanon, 1948). The same happens when we give weights to word vectors. We give more weights to events which occur by surprise and less weight to events which are expected. The most popular weighting concept in this domain is the idea of tf-idf which is explained later. \cite{Salton:88} defined a large family of tf-idf weighting functions and evaluated them on information retrieval tasks, demonstrating that tf-idf weighting can yield significant improvements over raw frequency. Term weighting is also significant in reducing the weights of highly correlated words such as \emph{hostage} and \emph{hostages} when they occur together (Church, 1995). \cite{Chirawichitchai:14} have found use of term weighting in sentiment classification in Thai language.

Lastly, a single learning algorithm in any domain cannot always induce the most accurate learner and hence we include in this work, the concept of ensemble learning though it is not ensemble in absolute sense. We require that our classifier should have properties of both a discriminative model and a generative classifier and hence merged the predictions of both achieving better results(majority voting). Since our main focus is not ensemble, so we donot look into other ensemble methods in detail.

\subsection{Sentiment Analysis}
In order to evaluate the efficacy of the model, we apply it to the task of sentiment analysis. Here the problem is that of identifying the polarity of sentences (Liu et al. 2012); for example: 
\begin{itemize}
\item Positive: {\dn rA\8{m} n\? khAnF kF r\327wtAr khF{\qva} Tmn\? nhF{\qva} dF} [Ramu didn't allow the pace of the story to subside]
\item Negative: {\dn pd\?{\qvb} pr EdKAyA jA rhA KO\327w Esn\?mAGr m\?
nhF{\qva} psr pAtA} [The horror shown on the screen didn't reach the theater]
\end{itemize}

This is a problem that has attracted reasonable attention in both Hindi and English (see section~\ref{sec:related}). There have been heuristic based and machine learning based models used in this domain~\cite{Wang:14}. Heuristic based methods, in general, classify text sentiments on the basis of total number of derived positive or negative sentiment oriented features. For example, Hatzivassiloglou and McKeown \cite{Hatzivassiloglou:97} described that adjectives are more predictive of sentiment classification and they predicted the sentiment of adjectives by inspecting them in conjunction with "and", "or", "but", "either/or" and "neither/nor". However the problem with this approach is that even if adjectives  are important in many languages but this model underestimates some predictive words of other parts-of-speech which may be important in some languages. Also these models rely heavily on human engineered features which, in general, is a domain and language dependent task.

In this work, we first learn a distributional word vector model as well as sentence vector model based on the wikipedia corpus as well as the sentiment corpus, and then we use this to discern the polarities on the existing corpora of movie and product reviews. To our own surprise, we find that even a simple additive composition model improves the state of the art in this task significantly (a gain of nearly 10\%). With weighted additive composition, the gain is more than 2\% beating the state-of-the-art in Hindi~\cite{Singh:15}. When used for the much better-researched, larger datasets of English the system does respectably, but well behind the very best models that attempt more complex composition models. So the question arises as to whether the very significant gains in Hindi are due to some quirk in the dataset, or could it be that Hindi word vectors are particularly informative,
e.g. owing to more highly inflected nature of its surface forms.  Also, if the results are not corpus-specific, it also raises the possibility that word vector methods may result in significant gains in
other similar problems for Hindi. 

For example, the sentences below give a brief idea of what positive and negative sentiment means.
\begin{itemize}
\item Positive: {\dn yh EPSm aQCF h\4}
\item Negative: {\dn yh smAn b\7{h}t KrAb h\4}
\end{itemize}
\footnotetext{en.wikipedia.org/wiki/List\_of\_languages\_by\_number\_of\_native\_speakers}
Majority of the existing work in this field is in English (Pang and Lee, 2008). \cite{Sharma:14} give a summary of work done in Hindi in the field of opinion mining.

We then go a step ahead to build a model using document vectors along with tfidf and ensemble techniques to achieve state-of-the-art result on IMDB movie review dataset achieving a significant improvement over the previous best. We have got an accuracy of 94.19\% by using an ensemble method merged with recurrent neural network to show the excellency of generative models merged with discriminative ones.
Our model uses concatenation of average word vectors, sentence vectors and tf-idf to build informative document vectors which constitute a feature set for our SVM classier. We have used skip-gram model (Mikolov et al., 2013b) for building word vectors because deep network model of Collobert et al.(2008) takes too much time for training. Also the word embeddings obtained by them are not as good as those obtained by Mikolov et al.(2013b). \\
For Hindi, we have experimented with tf-idf and weighted average word vectors for building our feature set. Our model shows slight improvement in performance if we filter out certain corpus based stop words. This is a first attempt to use word vectors for sentiment analysis in Hindi. Word vectors capture both semantic and syntactic information for a given corpus. Words which are semantically and syntactically related tend to be closer in high-dimensional space.\\

\section{The Problem}
The problem undertaken in this thesis is to use Distributional Semantics for under-resourced languages such as Hindi to cater to a very significant NLP task of sentiment analysis. We also try to highlight the differences in language and how efficacy of a model changes with language. We also took the task of using distributional semantics and general semantics with various forms of composition of vectors and even features to build good models for English as well. We have also laid a systematic background to the problem of context size and how it is related to syntactic and semantic similarity in text. 

\section{Contributions of this Thesis}
There are three main contributions of this thesis:
\begin{itemize}
\item We have successfully applied distributional models for Hindi and achieved state-of-the-art results on standard corpus of product and movie reviews. As a result, we have successfully published our work in regICON-2015 organized at IIT-BHU.
\item We have also compiled a corpus in Hindi containing around 700 Movie Reviews which is the largest corpus in Hindi in this domain. For reference,we will call it as \emph{700-Movie reviews} dataset.
\item We have achieved state-of-the-art results on IMDB Movie Review dataset(94.19\%) and have beaten the previous best by over 1.6\% in terms of accuracy.
\end{itemize}

\section{Organization of this Thesis}
The rest of this thesis is organized as follows. Chapter \ref{sec:related} presents related work done and discusses the various background work that one must be acquainted with in order to understand the work presented. Chapter \ref{sec:data} discusses the data sources and presents the statistics of each dataset. Chapter \ref{sec:dimensionality_reduction} gives an overview of dimensionality reduction techniques and how we have successfully applied them on vectors of our Hindi corpus to achieve very good results. Chapter \ref{sec:experiment} then discusses in detail the implemented methods and algorithms. Chapter \ref{sec:result} presents a summary of the results that was achieved. Chapter \ref{sec:conclusion_future_work} presents concluding remarks to our work and talks about what can be done further.
