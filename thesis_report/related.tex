\chapter{Related Work}
\label{sec:related}
Sentiment analysis is a well-known research area in NLP today (see reviews in~\cite{Liu:12} and Pang et al. (2008), and also
challenge in SemEval-2014).  Early work on movie review sentiments achieved an accuracy of 87.2\% (Pang et al. 2004) on a dataset that discarded objective sentences and used text categorization techniques
on the subjective sentences. Le and Mikolov (2014) use word vector models and obtain 92.6\% accuracy on IMDB movie review dataset.
They used distributed bag-of-words model, which they call as \emph{paragraph vector}. More difficult challenges involve short texts with nonstandard vocabularies,as in twitter.  Here, some authors focus on building extensive feature sets (e.g. Mohammad et al.(2013); F-score 89.14). \\
% WANG IS NOT BEING REFERRED TO
Wang et al. (2014) propose a word vector neural-network model, which takes both sentiment and semantic information into account. This word vector expression model learns word semantics and sentiment at the same time as well as fuses unsupervised contextual information and sentence level supervised labels.

There has been limited work on sentiment analysis in Hindi -- see review in
~\cite{Medagoda:13}, who surveys sentiment analysis in non-English languages). Joshi et al. (2010) compared three approaches: In-language sentiment
analysis, Machine Translation and Resource Based Sentiment Analysis. By using WordNet linking, words in English SentiWordNet were replaced by equivalent Hindi words to get H-SWN. The final accuracy achieved by them is 78.1\%.

\cite{Bakliwal:12}
traversed the WordNet ontology to antonyms and synonyms 
to identify polarity shifts in the word space. Further
improvements were achieved by using a partial stemmer (there is no good
stemmer / morphological analyzer for Hindi), and focusing on 
adjective/adverbs (45 + 75 seed words given to the system); their 
final accuracy was 79.0\% for the product review dataset. 
Mukherjee et al. (2012) presented the inclusion of discourse markers in a bag-of-words model and how it improved the sentiment classification accuracy by 2-4\%.  % <--- NOT HINDI
Mittal et al. (2013) incorporate hand-coded rules dealing with negation and discourse relations and extend the HSWN lexicon with more opinion words.  Their algorithm achieves  80.2\%
accuracy on classification of movie reviews on a separate dataset.\\

\cite{Mikolov:10} is a work The model has developed into
More details in \ref{sec:rnn}
