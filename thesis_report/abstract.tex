\cleardoublepage

\begin{center}
	\huge{\textbf{Abstract}}
\end{center}

In recent years, distributional semantics or vector models for words and documents have been proposed to capture both the syntactic and semantic similarities. Since these can be obtained in an unsupervised manner, they are of interest for under-resourced languages such as Hindi.  We test the efficacy of such an approach for Hindi, first by a subjective overview which shows that a reasonable measure of word similarity seems to be captured quite easily.  We then apply it to the sentiment analysis for two small Hindi databases from earlier work and our own built dataset of Hindi movie reviews. We also propose that dimensionality reduction techniques such as ANOVA-F and PCA are of great help to reduce noise and boost accuracy of our models.

In order to handle larger strings from the word vectors, several methods - additive, multiplicative, or tensor neural models, have been proposed.  Here we propose weighted additive average technique, which results in an impressive accuracy gain on state of the art by 12\% (from 80\%) for two review datasets.  The results suggest that it may be worthwhile to explore such methods further for Indian languages.\\
We have gone a step ahead with document vectors and built new features merged with various models to achieve state-of-the art results in sentiment classification on classical IMDB movie review dataset achieving an accuracy of 94.19\%(improvement of 1.61\% over previous best). We also implement an ensemble model which boosts our accuracy by about 0.5\% using recursive neural network.
