\chapter{Conclusion and Future Work}
\section{Inference}
		In this work we present an early experiment on the possibilities of distributional semantic models (word vectors) for low-resource, highly inflected languages such as Hindi.  What is interesting is that our word vector averaging method along with tf-idf results in improvements of accuracy compared to existing state-of-the art methods for sentiment analysis in Hindi (from 80.2\% to 89.9\%).
Distributional semantics approaches remain relatively under-explored for Indian languages, and our results suggest that there may be substantial benefits to exploring these approaches for Indian languages.  While this work has focussed on sentiment classification, it may also improve a range of tasks from verbal analogy tests to ontology learning, as has been reported for other languages.
In our future work, we seek to explore various compositional models - a) weighted average - where weights are determined based on cosine distances in vector space;  b) multiplicational models. Another aspect we are considering is to incorporate multiple word vectors for the same surface token in cases of polysemy - this would directly be useful for word sense disambiguation.  Identifying morphological variants would be another direction to explore for better accuracy. With regard to sentiment analysis, the idea of aspect-based models (or part-based sentiment analysis), which looks into constituents in a document and classify their sentiment polarity separately, remains to be explored in Hindi. Another point to note is that we are re-computing the word vectors
for the two review corpora, which are extremely small.  We may expect better performance  with a larger sentiment corpus.
%?? perhaps ASPECT requires parsing? 

We also observe that pruning high-frequency stop words improves the accuracy by around 0.45\%. This is most likely  because such words tend to occur in most of the documents and don't contribute to sentiment.  Similarly, words with very low frequency are noisy and can be pruned. For example, the word {\dn EPSm} occurs in 139/252 documents in Movie Review Dataset(55.16\%) and has little effect on sentiment.
Similarly words such as {\dn Es\388wAT\0} occur in 2/252 documents in Movie Review Dataset(0.79\%). These words don't provide much information.\\ 

Before concludiong, we return to the unexpectedly high improvement in accuracy achieved. One possibility we considered is that when the skip-grams are learned from the entire review corpus, it incorporates some knowledge of the test data.  But this seems unlikely since the difference in including this vs not including it, is not too significant.  The best explanation may be that the earlier methods, which were all in some sense based on a sentiWordnet, and at that one that was initially translated from English, were essentially very weak.  This is also clear in an analysis from
~\cite{Bakliwal:12}, which shows intern-annotator agreement on sentiment words are very poor (70\%) - i.e. about 30\% of these words have poor human agreement. Compared to this, the word vector model  
provides considerable power, especially as amplified by the tf-idf process. Thus, this also seems to underline the claim that distributional semantics is a topic worth exploring for Indian languages. 

\section{Improvements}
	
\section{Summary}
\begin{itemize}
	\item Corpus: This plays a very significant role in the process of sentiment prediction and has not been very widely tapped. Some corpus tend to be more focused towards sentiment while some contain a lot of information which is redundant as far as their sentiment score is concerned.
	\item Features: This parameter has been part of what we call in machine learing as \emph{Feature Engineering} and has been very popular whenever a new algorithm appears in the domain of Machine Learning or Natural Language Processing.
\end{itemize}
